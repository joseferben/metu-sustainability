\section{Gesellschaft}\label{sec:society}

Der gesellschaftliche Einfluss auf die Nachhaltigkeit des Anstiegs der
prognostizierten Tantalproduktion ist global Verteilt. Im Rahmen dieser Analyse
wird der Schwerpunkt auf die wichtigsten Herkunftslaender für die
Tantalproduktion relevanten Mineralien gelegt.

% TODO Figure?

\subsection{Indikatoren}

Nachfolgend werden jeweils die einzelnen Indikatoren beschrieben, welche die
Basis für die gesellschaftliche Entwicklung der Nachhaltigkeit bilden.

\subsubsection{Soziale Sicherheit}

Unter der sozialen Sicherheit werden Faktoren im Zusammenhang mit der
politischen Situation der Herkunftslaender zusammengefasst. Insbesondere der
Beitrag an die innenpolitische Stabilitaet des Landes traegt massgeblich zur
gesellschaftlichen Nachhaltigkeit bei.

\subsubsection{Gesundheit}

Die Arbeitsverhaeltnisse in den Produktionsstaetten für Tantal und die Folgen der
Umweltverschmutzung resultierend aus der Produktion auf die lokale Bevölkerung
sind die wichtigsten Aspekte für den Indikator Gesundheit.

\subsubsection{Solidaritaet}

Der Indikator zur Solidaritaet umfasst das Bewusstsein der Konsumenten und
Maerkte zur Herkunft und Produktion von Tantal.

\subsubsection{Chancengleichheit}

Die Chancengleichheit umfasst das Verhaeltniss von Arbeitnehmer und Arbeitnehmer
in den Produktionsstaetten, sowie den Einfluss auf die lokale Bevölkerung.
% Bergbau als Mittel gegen Armut.  Bei hoher Nachfrage höhere Preise ->
% Australien steigt wieder ein -> Arbeitsplaetze.

\subsection{Situation 2013}

% Finanzierung von Konfliktparteien.  Kongo ist einer der Hauptproduzenten ->
% ergo Konfliktmineral -> schlecht für Nachhaltigkeit Rwanda ist ein weiter
% Hauptproduzent und Nachbar aber wesentlich stabiler unteranderem dank einem
% stabileren politischen Klima und erfolgreichen Programmen TODO find explicit
% programms and cite them.

% Gesundheit der Minenarbeiter.  Folgen von möglicher Umweltverschmutzung.
% Statistiken zu Unfallraten in der Tantalproduktion.

% Offizielle Massnahmen  z.B. EU etc.

\subsection{Situation 2035}

\subsubsection{Gleicher Tantalverbrauch}

\subsubsection{Alternative Technologie}


@misc{USGSMine8:online, author = {Désirée E. Polyak}, title = {USGS Minerals
Information: Niobium (Columbium) and Tantalum}, howpublished =
{\url{https://minerals.usgs.gov/minerals/pubs/commodity/niobium/}}, month = {},
year = {}, note = {(Accessed on 09/13/2018)} }

