\section{Motivation & Methodik}\label{sec:motivation}

\subsection{Was sind Konfliktmineralien?}

In vielen Alltagsgegenständen wie Smartphones, Laptops oder Glühbirnen stecken Tantal, Wolfram, Zinn und Gold. Diese Metalle werden aus abgebauten Mineralien gewonnen. Förderung und Handel solcher Mineralien in Konfliktgebieten können zu schweren Menschenrechtsverletzungen und Verletzungen des humanitären Völkerrechts führen. Oftmals bleibt der Reichtum nicht in den Förderländern.
~\cite{definiti26:online}

\subsection{Methodik}

Nachfolgend untersuchen wir Konfliktmineralien im Kontext der Nachhaltigkeit. Wir betrachten die Veränderung der Auswirkungen von Abbau und Handel während der letzten 20 Jahre.

Dazu identifizieren wir Minerale, welche grossen Einfluss auf Wirtschaft, Umwelt und Gesellschaft haben. Wir wählen das Mineral mit dem grössten Einfluss aus und behandeln es detailliert. Es werden für jeden der drei Aspekte Wirtschaft, Umwelt und Gesellschaft Indikatoren gewählt und deren Veränderung in den letzten zwei Jahrzenten betrachtet.

Die Indikatoren der drei Aspekte fliessen in eine Gesamtbewertung ein. Diese soll zeigen, ob der Abbau und Handel von Konfliktmineralen nachhaltig stattfinden kann.
